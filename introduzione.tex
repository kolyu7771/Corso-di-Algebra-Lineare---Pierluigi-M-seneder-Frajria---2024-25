\chapter*{Introduzione}
L'algebra lineare è un argomento la cui complessità a che fare con la sua astrattezza e l'interconnessione tra gli argomenti: la comprensione profonda di certi argomenti, alcuni dei quali verranno affrontati relativamente presto, non può che avvenire molto più tardi nel corso.

Affrontare il corso in maniera ``compartimentalizzata'' è un approccio fallimentare, in quanto è impossibile dividere nettamente un argomento dall'altro.

Il professor Möseneder ha strutturato il corso in quattro sezioni:
\begin{enumerate}
    \item Parte computazionale: vettori e matrici;
    \item Parte astratta: spazi vettoriali;
    \item Parte di applicazione: diagonalizzazione;
    \item Parte di applicazione: geometria.
\end{enumerate}

Per cercare di rendere più chiara la motivazione dello studio di certi argomenti verranno fatti cenni di applicazioni per argomenti futuri, con l'obiettivo di chiarirne l'importanza.