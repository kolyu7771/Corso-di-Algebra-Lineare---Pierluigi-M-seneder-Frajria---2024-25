\chapter{Determinante}
\section{Definizione}
La definizione di determinante è una difficile da dare in questo momento, come è difficile in questo momento rendersi conto di quanto sia uno strumento potente. Una definizione adeguata di determinante, a questo punto, sarebbe la seguente:
\begin{newdef}{Determinante}
    Il \textbf{determinante} è un valore caratteristico di una matrice \textit{quadrata}, che ne esprime diverse proprietà algebriche e geometriche.

    Lo si indica con:
    \[
        \det A \text{ o } |A|
    \]
\end{newdef}
La definizione, come facile notare, è abbastanza vaga. ``In che senso è un \textit{valore caratteristico}?''; ``Quali proprietà esprime? Come?''.

Purtroppo le risposte a queste domande riceveranno risposta solo più avanti.

Alcune applicazioni del determinante sono il calcolo dell'inversa di una matrice, del rango di una matrice, che sarà introdotto propriamente nella parte ``astratta'' del corso, e delle soluzioni di sistemi lineari crameriani; inoltre avrà un ruolo fondamentale nel problema della diagonalizzazione, argomento principe della terza parte del corso.

\section{Determinanti di matrici di ordine 1, 2 e 3}
Il determinante di una matrice $1 \times 1$ è banale.

\begin{teo}{Determinante di una matrice di ordine 1}
    Il determinante di una matrice di ordine 1 è l'unico elemento della matrice.
    \[
        \det[a] = a
    \]
\end{teo}

Invece il determinante di una matrice $2 \times 2$ ha una leggermente formula più complessa:
\begin{teo}{Determinante di una matrice di ordine 2}
    Il determinante di una matrice di ordine 2 è la differenza tra il prodotto degli elementi sulla diagonale discendente (\textit{diagonale principale})e il prodotto degli elementi sulla diagonale ascendente (\textit{antidiagonale}).
    \[
        \det
        \begin{sqmatrix}{c c}
            a & b \\
            c & d
        \end{sqmatrix}
        = ad - bc
    \]
\end{teo}
\begin{esempio}
    \textbf{Calcolare il determinante della seguente matrice di ordine 2:}
    \[
        A =
        \begin{sqmatrix}{c c}
            6 & 3 \\
            7 & 6
        \end{sqmatrix}
    \]
    Utilizzando la formula proposta qua sopra si ottiene:
    \[
        \det A = 6 \cdot 6 - 3 \cdot 7 = 36 - 21 = 15
    \]
\end{esempio}
\begin{exer}
    \textbf{Calcolare il determinante delle seguenti matrici di ordine 2:}

    $
        \begin{sqmatrix}{c c}
            4 & 2 \\
            4 & 0
        \end{sqmatrix}
    $;

    $
        \begin{sqmatrix}{c c}
            3 & 0 \\
            1 & 3
        \end{sqmatrix}
    $;

    $
        \begin{sqmatrix}{c c}
            3 & 9 \\
            5 & 15
        \end{sqmatrix}
    $;
\end{exer}

Quando si arriva all'ordine 3 la formula si complica ancora di più:
\begin{multline*}
    \det
    \begin{sqmatrix}{c c c}
        a_{11} & a_{12} & a_{13} \\
        a_{21} & a_{22} & a_{23} \\
        a_{31} & a_{32} & a_{33} \\
    \end{sqmatrix}
    = a_{11}a_{22}a_{33} - a_{11}a_{23}a_{32} - a_{12}a_{21}a_{33} \\ - a_{13}a_{22}a_{31} + a_{12}a_{23}a_{31} + a_{13}a_{21}a_{32}
\end{multline*}
Ovviamente ricordarsi una formula del genere è improponibile. Fortunatamente esiste un metodo di calcolo del determinante di una matrice $3 \times 3$ relativamente semplice.

\begin{teo}{Regola di Sarrus}
    Per calcolare il determinante di una matrice di ordine 3 la si riscrive, aggiungendo dopo l'ultima colonna le prime due, e sommando i prodotti sulle 3 diagonali discendenti e sottraendo i prodotti sulle 3 diagonali ascendenti.
    \begin{center}
        \tikzset{external/export=true}
        \begin{tikzpicture}
            \foreach \i [evaluate=\i as \ip using int(\i+1)] in {0,1,2}{
                    \foreach \j [evaluate=\j as \jp using int(-\j+1)] in {0,-1,-2}{
                            \node at (\i,\j) {$a_{\ip \jp}$};
                        }
                }
            \foreach \i [evaluate=\i as \ip using int(\i-2)] in {3,4}{
                    \foreach \j [evaluate=\j as \jp using int(-\j+1)] in {0,-1,-2}{
                            \node at (\i,\j) {$a_{\ip \jp}$};
                        }
                }
            \foreach \i/\j in {0.5/-0.5,1.5/-0.5,1.5/-1.5,2.5/-0.5,2.5/-1.5,3.5/-1.5}{
                    \node at (\i,\j) {$\searrow$};
                }
            \foreach \i/\j in {0.5/-1.5,1.5/-0.5,1.5/-1.5,2.5/-0.5,2.5/-1.5,3.5/-0.5}{
                    \node at (\i,\j) [Red] {$\nearrow$};
                }
        \end{tikzpicture}
        \tikzset{external/export=false}
    \end{center}
    \[
        a_{11}a_{22}a_{33} + a_{21}a_{32}a_{13} + a_{31}a_{12}a_{23} \textcolor{Red}{- a_{13}a_{22}a_{31} - a_{23}a_{32}a_{11} - a_{33}a_{12}a_{21}}
    \]
\end{teo}
\begin{esempio}
    \textbf{Calcolare il determinante della seguente matrice di ordine 3:}
    \[
        A =
        \begin{sqmatrix}{c c c}
            1 & 0 & 2 \\
            1 & 6 & 0 \\
            0 & 7 & 2
        \end{sqmatrix}
    \]
    Usando la regola di Sarrus si ottiene:
    \[
        \det A = 1 \cdot 6 \cdot 2 + 0 \cdot 0 \cdot 0 + 2 \cdot 1 \cdot 7 - 0 \cdot 6 \cdot 2 - 7 \cdot 0 \cdot 1 - 2 \cdot 1 \cdot 0 = 12 + 14 = 26
    \]
\end{esempio}
\begin{exer}
    \textbf{Calcolare il determinante della seguente matrice di ordine 3:}
    
    $
        \begin{sqmatrix}{c c c}
            4 & 2 & 2\\
            4 & 0 & 9 \\
            2 & 1 & 1
        \end{sqmatrix}
    $;

    $
        \begin{sqmatrix}{c c c}
            3 & 10 & 0\\
            2 & 0 & 1 \\
            2 & 1 & 0
        \end{sqmatrix}
    $;

    $
        \begin{sqmatrix}{c c c}
            0 & 1 & 0\\
            7 & 0 & 0 \\
            21 & 100 & 0
        \end{sqmatrix}
    $;
\end{exer}

\section{Determinanti di matrici di ordine superiore}
Come si può immaginare, le formule dei determinanti di matrice di ordine superiore diventano sempre più complesse. Infatti la formula del determinante di una matrice di ordine $n$ ha ben $n!$ elementi! Naturalmente pensare di ricordarsi le formule di questi determinanti è pura follia.

Esiste, fortunatamente, un \textit{algoritmo} ricorsivo che permette di calcolare il determinante di una qualsiasi matrice di ordine $n$ sapendo fare il determinante per le matrici di ordine $n - 1$. Questo algoritmo è detto \textbf{sviluppo di Laplace}, ma prima di affrontarlo è necessario introdurre il concetto di \textit{minore complementare} e \textit{complemento algebrico}.

\begin{newdef}{Minore complementare}
    Il \textbf{minore complementare} di una matrice quadrata $M_{i;j}$ è il minore ottenuto rimuovendo l'$i$-esima riga e la $j$-esima colonna dalla matrice di partenza.

    \[
        A = 
        \begin{sqmatrix}{c c c}
            3 & 5 & 1 \\
            1 & 3 & 3 \\
            0 & 2 & 5
        \end{sqmatrix}
        \longrightarrow
        M_{2;2} =
        \begin{sqmatrix}{c c}
            3 & 1 \\
            0 & 5
        \end{sqmatrix}
    \]
\end{newdef}
\begin{newdef}{Complemento algebrico}
    Il \textbf{complemento algebrico}, o \textbf{cofattore}, di un elemento $a_{ij}$ della matrice quadrata $A$ è il determinante del minore complementare $M_{i;j}$ con anteposto un segno $+$ se la somma degli indici è pari e $-$ se è dispari.
    \[
        A_{ij} = (-1)^{i + j}\det M_{i;j}
    \]
\end{newdef}
\begin{nb}
    Invece di dover calcolare ogni volta la somma e controllare se anteporre un $+$ o un $-$ si può notare l'alternanza nei segni e costruirsi la seguente matrice:
    \[
        \begin{sqmatrix}{c c c c}
            + & - & + & \dots \\
            - & + & - & \dots \\
            + & - & + & \dots \\
            \vdots & \vdots & \vdots & \ddots
        \end{sqmatrix}
    \]
\end{nb}

A questo punto è possibile introdurre lo sviluppo di Laplace.
\begin{teo}{Sviluppo di Laplace}
    Sia $A$ una matrice quadrata di ordine $n$. Allora si ha che, fissata una riga $i$:
    \[
        \det A = \sum_{j = 1}^n a_{ij}A_{ij}
    \]
    Parimenti, fissando una colonna $j$ si ha:
    \[
        \det A = \sum_{i = 1}^n a_{ij}A_{ij}
    \]
\end{teo}
\begin{nb}
    Quando si usa lo sviluppo di Laplace, è bene scegliere la riga/colonna con il maggior numero di zeri.
\end{nb}
\begin{esempio}
    \textbf{Calcolare il determinante della seguente matrice:}
    \[
        A =
        \begin{sqmatrix}{c c c c}
            1 & 2 & -1 & 1 \\
            2 & 0 & 1 & 0 \\
            1 & 1 & 1 & 1 \\
            0 & 2 & 1 & 0
        \end{sqmatrix}
    \]
    Per lo sviluppo di Laplace si può scegliere la quarta riga o la quarta colonna. In questo caso verrà scelta la quarta colonna. Applicando la formula si ha:
    \[
        \det A = 1 \cdot (-1)^{4 + 1} \cdot \det
        \begin{sqmatrix}{c c c}
            2 & 0 & 1 \\
            1 & 1 & 1 \\
            0 & 2 & 1
        \end{sqmatrix}
        + 1 \cdot (-1)^{4 + 3} \cdot \det
        \begin{sqmatrix}{c c c}
            1 & 2 & -1 \\
            2 & 0 & 1 \\
            0 & 2 & 1
        \end{sqmatrix}
    \]
    Applicando la regola di Sarrus si ottiene:
    \[
        = (-1)(0) + (-1)(-10) = 10
    \]
\end{esempio}

\section{Proprietà del determinante}
Per il determinante valgono le seguenti proprietà:
\begin{teo}{Proprietà del determinante}
    \begin{enumerate}
        \item Il determinante della matrice quadrata identità è 1:\\ 
            $\det I_n = 1 \;\forall n > 0$;
        \item Il determinante di una matrice e della sua trasposta sono uguali:\\
            $\det A = \det A^T$;
        \item Se $A$ è una matrice triangolare allora il suo determinante è dato dal prodotto degli elementi sulla sua diagonale:\\
        $\det A = \prod_{i = 1}^n a_{ii}$;
        \item Se si scambiano due righe/colonne di $A$ il suo determinante cambia di segno;
        \item Se si moltiplica una sua riga di $A$ per una costante $k$ il suo determinante verrà moltiplicato per $A$;
        \item Se si fa la somma tra multipli di due righe di $A$ il determinante resta invariato;
        \item Se una riga/colonna è nulla oppure due righe/colonne sono uguali o proporzionali allora il determinante è nullo;
        \item Formula di Binet: Il determinante del prodotto è il prodotto dei determinanti:\\
            $\det(AB) = \det A \det B$;
        \item Se $A$ è una matrice a blocchi triangolare superiore/inferiore o diagonale, allora il suo determinante è dato dal prodotto dei determinanti dei blocchi sulla diagonale.
    \end{enumerate}
\end{teo}

