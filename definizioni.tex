\raggedbottom
\pagestyle{fancy}
\fancyhf{}
\fancyhead[LE,RO]{\thepage}
\fancyhead[RE]{\nouppercase{\leftmark}}
\fancyhead[LO]{\nouppercase{\rightmark}}
\fancyfoot[C]{-~\thepage~-}

\fancypagestyle{plain}{
    \fancyhf{}
    \fancyfoot[C]{\thepage}
    \renewcommand{\headrulewidth}{0pt}
}

\renewcommand{\chaptermark}[1]{\markboth{Capitolo \thechapter\ -\ #1}{}}
\renewcommand{\sectionmark}[1]{\markright{\thesection\ #1}}

\newcommand{\chapterheader}[1]{
	\begin{tikzpicture}[baseline=(current bounding box.north), remember picture, overlay]
		\filldraw[Red] (0,0) rectangle (2,2);
		\filldraw[Maroon] (0,0) -- (0.5,-0.5) -- (0.5,0);
		\node[white, font=\Huge\bfseries] at (1,1) {\thechapter};
		\filldraw[Red] (2,2) rectangle (\textwidth, 1);
		\filldraw[Maroon] (\textwidth,2) -- (\textwidth - 0.5cm,2.5) -- (\textwidth-0.5cm,1.5) -- (\textwidth,1);
		\node[anchor=west, white, font=\Large\bfseries] at (2,1.5) {#1};
	\end{tikzpicture}
}

\newcommand{\sectionheader}[1]{
	\begin{tikzpicture}[baseline=(current bounding box.north), remember picture, overlay]
		\filldraw[Emerald] (0,-0.05) rectangle (1.75,0.55);
		\node[white, font=\Large\bfseries] at (0.875,0.25) {\thesection};
		\node[anchor=west, Red, font=\Large\bfseries] at (1.8,0.25) {#1};
	\end{tikzpicture}
}

\titleformat{\chapter}[display]
  {\sffamily\Huge\bfseries}
  {}
  {0pt} % Spaziatura tra etichetta e titolo
  {
    \vspace{-4cm}
    \chapterheader{#1}
  }
  
\titleformat{name=\chapter,numberless}[display]
  {\sffamily\Huge\bfseries}
  {}
  {0pt} % Spaziatura tra etichetta e titolo
  {#1}

\titleformat{\section}[display]
  {\sffamily\Large\bfseries}
  {}
  {0pt} % Spaziatura tra etichetta e titolo
  {
    \vspace{-0.75cm}
    \sectionheader{#1}
  }

\titleformat{\subsection}[display]
  {\sffamily\large\bfseries}
  {}
  {0pt} % Spaziatura tra etichetta e titolo
  {
    \vspace{-0.5cm}
    \sffamily\bfseries\textcolor{Red!75!White}{#1}\rmfamily\mdseries
  }

\tcbset{MyStyle/.style={
    tile,
    breakable,
    rounded corners=downhill,
    drop fuzzy shadow = Blue,
    arc=4mm,
    fonttitle=\bfseries\sffamily
}}

\newtcolorbox[auto counter, number within=section]{teo}[1]{
    MyStyle,
    colback=Purple!15!white,
    colbacktitle=Purple!85!white,
    title = Teorema \thetcbcounter: #1
}

\newtcolorbox{newdef}[1]{
    MyStyle,
    colback=Blue!15!white,
    colbacktitle=Blue!85!white,
    title = Definizione: #1
}

\newtcolorbox{nb}{
    MyStyle,
    colback=Red!15!white,
    colbacktitle=Red!85!white,
    title = Nota bene!
}

\newtcolorbox[auto counter, number within=section]{exer}{
    MyStyle,
    colback=Orange!15!white,
    colbacktitle=Orange!85!white,
    title = Esercizio \thetcbcounter
}

\newtcolorbox[auto counter, number within=section]{esempio}{
    MyStyle,
    colback=Magenta!15!white,
    colbacktitle=Magenta!85!white,
    title = Esempio \thetcbcounter
}

\newtcolorbox{newdim}{
    MyStyle,
    colback=Green!15!white,
    colbacktitle=Green!85!white,
    title = Dimostrazione
}

\newenvironment{sqmatrix}
{\left[\begin{array}}
{\end{array}\right]}

\newenvironment{sqcolvec}
{\left[\begin{array}{c}}
{\end{array}\right]}
  
\newenvironment{exttikzpicture}{\tikzset{external/export=true}\begin{tikzpicture}}{\end{tikzpicture}\tikzset{external/export=false}}

\newcommand{\qedwhite}{\hfill\square}