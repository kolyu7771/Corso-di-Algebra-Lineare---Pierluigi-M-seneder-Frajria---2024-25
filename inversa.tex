\chapter{Inversa di una matrice}
\section{Definizione}
Nell'insieme dei numeri reali ogni numero ha un inverso, ovvero:
\[
    \forall a \in \mathbb{R} \; \exists a^{-1} : aa^{-1} = a^{-1}a = 1
\]
Quindi per ogni numero reale esiste un altro numero che, moltiplicato con esso, ha come risultato 1.

Ora si risolverà il problema analogo sulle matrici: avendo già definito il concetto di potenza per le matrici ha senso chiedersi se per ogni matrice quadrata esiste un'altra matrice che, quando moltiplicata, dà come risultato la matrice identità.

\begin{newdef}{Matrice inversa}
    Una matrice $A \in \mathbb{R}^{n \times n}$ è detta \textbf{invertibile} se esiste $A^{-1} \in \mathbb{R}^{n \times n}$ tale che:
    \[
        AA^{-1} = A^{-1}A = I_n
    \]
\end{newdef}

Questo problema è fondamentale nel caso delle trasformazioni lineari, in quanto se la matrice associata alla trasformazione è invertibile anche la trasformazione lo sarà; a questo argomento, però, ci si arriverà a tempo debito.

\section{Calcolo tramite determinante}
Per calcolare il determinante in questo modo bisogna definire l'\textit{aggiunto classico} di una matrice.
\begin{newdef}{Aggiunto classico}
    L'\textbf{aggiunto classico} di una matrice $A \in \mathbb{R}^{n \times n}$ è la trasposta della matrice $C$ i cui elementi sono il complemento algebrico dei corrispondenti elementi di $A$.
    \[
        C =
        \begin{sqmatrix}{c c c c}
            A_{11} & A_{12} & \dots & A_{1n} \\
            A_{21} & A_{22} & \dots & A_{2n} \\
            \vdots & \vdots & \ddots & \vdots \\
            A_{n1} & A_{n2} & \dots & A_{nn}
        \end{sqmatrix}
    \]
    \[
        \text{agg}\,A =
        \begin{sqmatrix}{c c c c}
            A_{11} & A_{21} & \dots & A_{n1} \\
            A_{12} & A_{22} & \dots & A_{n2} \\
            \vdots & \vdots & \ddots & \vdots \\
            A_{1n} & A_{2n} & \dots & A_{nn}
        \end{sqmatrix}
    \]
\end{newdef}
\begin{teo}{Determinante con aggiunto classico}
    La matrice quadrata diagonale i cui elementi non nulli sono il determinante di $A$ è data dal prodotto di $A$ con il suo aggiunto classico; questo prodotto è anche commutativo.
    \[
        A\,\text{agg}\,A = \text{agg}\,(A)A = \det(A)I
    \]
\end{teo}
\begin{newdim}
    Sia $A\,\text{agg}\,A = B$. Si prenda l'elemento generico $b_{ij}$ della matrice. 
    
    Per $i = j$ si ha:
    \[
        b_{ii} = a_{i1}A_{i1} + a_{i2}A_{i2} + \dots + a_{in}A_{in} =
    \]
    \[
        = (-1)^{i + 1} a_{i1} \det M_{i;1} + (-1)^{i + 2} a_{i2} \det M_{i;2} + \dots + (-1)^{i + n} a_{in} \det M_{i;n} = 
    \]
    \[
        = \sum_{j = 1}^{n} a_{ij} \det M_{i;j} = \det A
    \]

    Per $i \neq j$ la questione è più complessa. Formalmente si ha:
    \[
        b_{ij} = a_{i1}A_{j1} + a_{i2}A_{j2} + \dots + a_{in}A_{jn} =
    \]
    \[
        = (-1)^{j + 1} a_{i1} \det M_{j;1} + (-1)^{j + 2} a_{i2} \det M_{j;2} + \dots + (-1)^{j + n} a_{in} \det M_{j;n}
    \]
    Questa può essere vista come lo sviluppo di Laplace del determinante di una matrice con due righe uguali, e perciò vale 0.

    Per poter intuirlo meglio si prenda il caso $3 \times 3$:
    \[
        A = 
        \begin{sqmatrix}{c c c}
            a_{11} & a_{12} & a_{13} \\
            a_{21} & a_{22} & a_{23} \\
            a_{31} & a_{23} & a_{33}
        \end{sqmatrix}
    \]
    I complementi algebrici $A_{11}, A_{12}, A_{13}$ sono rispettivamente:
    \[
        A_{11} = \det
        \begin{sqmatrix}{c c}
            a_{22} & a_{23} \\
            a_{32} & a_{33}
        \end{sqmatrix}
        \quad
        A_{12} = \det
        \begin{sqmatrix}{c c}
            a_{21} & a_{23} \\
            a_{31} & a_{33}
        \end{sqmatrix}
        \quad
        A_{13} = \det
        \begin{sqmatrix}{c c}
            a_{21} & a_{22} \\
            a_{31} & a_{32}
        \end{sqmatrix}
    \]
    Il secondo elemento di $B$ sarà dato da:
    \[
        a_{21}\det
        \begin{sqmatrix}{c c}
            a_{22} & a_{23} \\
            a_{32} & a_{33}
        \end{sqmatrix}
        -
        a_{22}\det
        \begin{sqmatrix}{c c}
            a_{21} & a_{23} \\
            a_{31} & a_{33}
        \end{sqmatrix}
        +
        a_{33}\det
        \begin{sqmatrix}{c c}
            a_{21} & a_{22} \\
            a_{31} & a_{32}
        \end{sqmatrix}
    \]
    (nb. il $(-1)^{i+j}$ è già stato preso in considerazione). Questo può essere visto come lo sviluppo di Laplace sulla prima riga di un'altra matrice, che verrà chiamata $M$. I complementi algebrici rispetto alla prima riga sono uguali a quelli di $A$, quindi le prime due righe saranno le stesse:
    \[
        M = 
        \begin{sqmatrix}{c c c}
            * & * & * \\
            a_{21} & a_{22} & a_{23} \\
            a_{31} & a_{23} & a_{33}
        \end{sqmatrix}
    \]
    I coefficienti dello sviluppo di Laplace sono $a_{21}, a_{22}, a_{23}$, quindi questi saranno gli elementi della prima riga:
    \[
        M = 
        \begin{sqmatrix}{c c c}
            a_{21} & a_{22} & a_{23} \\
            a_{21} & a_{22} & a_{23} \\
            a_{31} & a_{23} & a_{33}
        \end{sqmatrix}
    \]
    Avendo due righe uguali, si avrà che $\det M = 0$, e perciò ogni elemento della matrice $B$ che non si trova sulla diagonale principale sarà nullo.
    
    \hfill$\qedwhite$
\end{newdim}

Usando l'identità appena dimostrata è possibile trovare la formula per l'inversa della matrice tramite i seguenti passaggi:
\[
    \text{agg}\,(A)A = \det(A)I
\]
\[
    \frac{\text{agg}\,A}{\det A}A = I
\]
\[
    \frac{\text{agg}\,A}{\det A} = A^{-1}
\]
Per cui:

\begin{teo}{Formula dell'inversa con il determinante}
    L'inversa della matrice $A$ è data dal rapporto tra il suo aggiunto classico e il suo determinante:
    \[
        A^{-1} = \frac{\text{agg}\,A}{\det A}
    \]
\end{teo}
Questa formula implica direttamente il seguente criterio:
\begin{teo}{Criterio di invertibilità (1)}
    Una matrice è \textbf{invertibile} se e solo se il suo determinante non è nullo.
    \[
        \exists\,A^{-1} \iff \det A \neq 0
    \]
\end{teo}
\begin{esempio}
    \textbf{Determinare se la seguente matrice è invertibile e, in tal caso, calcolarne l'inversa.}
    \[
        A =
        \begin{sqmatrix}{c c c}
            1 & 0 & 1 \\
            1 & 1 & 1 \\
            1 & 1 & 0
        \end{sqmatrix}
    \]
    Tramite la regola di Sarrus si trova che:
    \[
        \det A = 0 + 0 + 1 - 1 - 1 - 0 = -1 \neq 0
    \]
    La matrice $A$ è invertibile.

    Per trovarne l'inversa si deve trovare l'aggiunto classico di $A$.
    \[
        \text{agg}\,A =
        \begin{sqmatrix}{c c c}
            A_{11} & A_{21} & A_{31} \\
            A_{12} & A_{22} & A_{32} \\
            A_{13} & A_{23} & A_{33}
        \end{sqmatrix}
    \]
    
    \[
        A_{11} = \left|
        \begin{array}{c c}
            1 & 1 \\
            1 & 0
        \end{array}
        \right|
        = -1
        \quad
        A_{12} = -\left|
        \begin{array}{c c}
            1 & 1 \\
            1 & 0
        \end{array}
        \right|
        = 1
        \quad
        A_{13} = \left|
        \begin{array}{c c}
            1 & 1 \\
            1 & 1
        \end{array}
        \right|
        = 0
    \]
    \[
        A_{21} = -\left|
        \begin{array}{c c}
            0 & 1 \\
            1 & 0
        \end{array}
        \right|
        = 1
        \quad
        A_{22} = \left|
        \begin{array}{c c}
            1 & 1 \\
            1 & 0
        \end{array}
        \right|
        = -1
        \quad
        A_{23} = -\left|
        \begin{array}{c c}
            1 & 0 \\
            1 & 1
        \end{array}
        \right|
        = -1
    \]

    \[
        A_{31} = \left|
        \begin{array}{c c}
            0 & 1 \\
            1 & 1
        \end{array}
        \right|
        = -1
        \quad
        A_{32} = -\left|
        \begin{array}{c c}
            1 & 1 \\
            1 & 1
        \end{array}
        \right|
        = 0
        \quad
        A_{33} = \left|
        \begin{array}{c c}
            1 & 0 \\
            1 & 1
        \end{array}
        \right|
        = 1
    \]
    
    \[
        \text{agg}\,A =
        \begin{sqmatrix}{c c c}
            -1 & 1 & -1 \\
            1 & -1 & 0 \\
            0 & -1 & 1
        \end{sqmatrix}
    \]
    Dividendo per il determinante si ottiene la matrice inversa:
    \[
        A^{-1} =
        \begin{sqmatrix}{c c c}
            1 & -1 & 1 \\
            -1 & 1 & 0 \\
            0 & 1 & -1
        \end{sqmatrix}
    \]
\end{esempio}

\begin{exer}
    \textbf{Determinare se le seguenti matrici sono invertibili e, in tal caso, calcolarne l'inversa:}
    
    $
        \begin{sqmatrix}{c c c}
            6 & 4 & 0 \\
            6 & 5 & 6 \\
            6 & 1 & 7
        \end{sqmatrix}
    $;

    $
        \begin{sqmatrix}{c c c}
            4 & 2 & 2 \\
            1 & 0 & 1 \\
            6 & 0 & 4
        \end{sqmatrix}
    $;

    $
        \begin{sqmatrix}{c c c c}
            3 & 4 & 5 & 1 \\
            3 & 1 & 1 & 3 \\
            5 & 4 & 2 & 0 \\
            -7 & -8 & 4 & 14
        \end{sqmatrix}
    $;
\end{exer}

\section{Calcolo tramite Gauss-Jordan}
Un altro metodo, che richiede meno calcoli per matrici più grandi, è quello di affiancare a destra della propria matrice la matrice identità e provare a portarla, attraverso operazioni elementari, a sinistra. Ovvero:
\begin{teo}{Calcolo dell'inversa con Gauss-Jordan}
    Affiancando a destra della matrice $A \in \mathbb{R}^{n \times n}$ la matrice identità $I_n$ e applicando il metodo di eliminazione di Gauss-Jordan si ottiene, al posto di $I$, $A^{-1}$.
    \[
        (A|I_n) \sim (I_n|A^{-1})
    \]
\end{teo}
Evidentemente il tutto può funzionare se e solo se nell'algoritmo non si ottengono righe di zeri, ovvero se e solo se ogni colonna di $A$ ha un pivot.
\begin{teo}{Criterio di invertibilità (2)}
    Una matrice è \textbf{invertibile} se e solo se la sua forma a gradini ridotta è uguale alla matrice identità.
    \[
        \exists\,A^{-1} \iff \text{rref}\,A = I_n
    \]
\end{teo}
\begin{esempio}
    \textbf{Determinare se la seguente matrice è invertibile e, in tal caso, calcolarne l'inversa:}
    \[
        A =
        \begin{sqmatrix}{c c c}
            4 & 7 & 2 \\
            0 & 2 & 2 \\
            4 & 1 & 4
        \end{sqmatrix}
    \]
    Si affianca ad $A$ la matrice identità, ottenendo:
    \[
        (A|I) =
        \begin{sqmatrix}{c c c c c c}
            4 & 7 & 2 & 1 & 0 & 0 \\
            0 & 2 & 2 & 0 & 1 & 0 \\
            4 & 1 & 4 & 0 & 0 & 1
        \end{sqmatrix}
    \]
    Facendo le seguenti operazioni elementari di riga ci si porta ad avere la seguente matrice:
    \begin{enumerate}
        \item $R_2 \leftrightharpoons R_3$
        \item $R_2 = R_1 - R_2$
        \item $R_3 = R_2 - 3 \cdot R_3$
        \item $R_2 = 4 \cdot R_2 - R_3$
        \item $R_1 = 4 \cdot R_1 + R_3$
        \item $R_1 = 6 \cdot R_1 - 7 \cdot R_2$
        \item $R_1 = \frac{R_1}{96}$
        \item $R_2 = \frac{R_2}{24}$
        \item $R_3 = -\frac{R_3}{8}$
    \end{enumerate}
    \[
        \begin{sqmatrix}{c c c c c c}
            4 & 7 & 2 & 1 & 0 & 0 \\
            0 & 2 & 2 & 0 & 1 & 0 \\
            4 & 1 & 4 & 0 & 0 & 1
        \end{sqmatrix}
        \sim
        \begin{sqmatrix}{c c c c c c}
            1 & 0 & 0 & \frac{3}{32} & -\frac{13}{32} & \frac{5}{32} \\[2pt]
            0 & 1 & 0 & \frac{1}{8} & \frac{1}{8} & -\frac{1}{8} \\[2pt]
            0 & 0 & 1 & -\frac{1}{8} & \frac{3}{8} & \frac{1}{8}
        \end{sqmatrix}
        = (I|A^{-1})
    \]
    Dato che le prime tre colonne della forma a gradini ridotta hanno un pivot la matrice è invertibile e l'inversa è:
    \[
        A^{-1} =
        \begin{sqmatrix}{c c c}
            \frac{3}{32} & -\frac{13}{32} & \frac{5}{32} \\[2pt]
            \frac{1}{8} & \frac{1}{8} & -\frac{1}{8} \\[2pt]
            -\frac{1}{8} & \frac{3}{8} & \frac{1}{8}
        \end{sqmatrix}
    \]
\end{esempio}
\begin{exer}
    \textbf{Determinare se le seguenti matrici sono invertibili e, in tal caso, calcolarne l'inversa:}
    
    $
        \begin{sqmatrix}{c c c}
            4 & 1 & 0 \\
            7 & 7 & 6 \\
            4 & 7 & 1
        \end{sqmatrix}
    $;

    $
        \begin{sqmatrix}{c c c}
            3 & 0 & 5 \\
            6 & 1 & 6 \\
            0 & 6 & 6
        \end{sqmatrix}
    $;

    $
        \begin{sqmatrix}{c c c}
            1 & 1 & 2 \\
            6 & 4 & 3 \\
            4 & 4 & 4
        \end{sqmatrix}
    $;
\end{exer}