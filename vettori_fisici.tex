\chapter{Vettori fisici}
Uno dei modi migliori di visualizzare i concetti spiegati nelle prossime parti è quello geometrico: se si è già studiata una materia come Fisica, si sarà già familiari con i vettori come ``frecce nello spazio'', ma è utile reintrodurre questi concetti e porre le basi della visualizzazione dei concetti dell'Algebra Lineare con l'uso dei vettori fisici.

\section{Definizione}
\begin{newdef}{Vettore fisico}
    Un \textbf{vettore fisico} è definito da 3 caratteristiche:
    \begin{enumerate}
        \item \textbf{Norma:} la sua lunghezza; è anche chiamata intensità;
        \item \textbf{Direzione:} la retta su cui giace;
        \item \textbf{Verso:} dove sta puntando.
    \end{enumerate}
\end{newdef}

Definendo il punto di partenza e il punto d'arrivo del vettore è possibile rappresentarlo come un segmento orientato:
\begin{center}
    \tikzset{external/export = true}
    \begin{tikzpicture}
        \node at (0,0) (P) {$P$};
        \node at (5,2) (Q) {$Q$};
        \draw[-Latex] (P) -- (Q);
    \end{tikzpicture}
    \tikzset{external/export = false}
\end{center}
\begin{nb}
    Il vettore nullo è rappresentato dal segmento orientato $\vec{PP}$.
\end{nb}
Questo segmento orientato è anche chiamato \textit{vettore libero}. Questo perché è possibile disegnare più segmenti orientati che descrivono lo stesso vettore, ovvero che hanno stessa norma, direzione e verso.
\begin{center}
    \tikzset{external/export = true}
    \begin{tikzpicture}
        \draw[-Latex] (0,0) -- (5,2);
        \draw[-Latex] (1,-2) -- (6,0);
        \draw[-Latex] (-2,1) -- (3,3);
    \end{tikzpicture}
    \tikzset{external/export = false}
\end{center}
Questi segmenti sono detti equipollenti.
\begin{newdef}{Equipollenza}
    Due segmenti orientati sono detti \textbf{equipollenti} se collegano due coppie di punti diverse e hanno stessa norma, direzione e verso.

    Un vettore libero è la \textbf{classe di equipollenza} di tutti i segmenti orientati che lo rappresentano.
\end{newdef}

\section{Somma e prodotto con scalare}
Le operazioni di somma e prodotto con scalare sono sempre le stesse e seguono le solite proprietà, ma è necessario comprendere come vengono svolte con i vettori liberi.
\subsection{Somma - Legge di Galileo}
Il primo modo di sommare i vettori liberi è la legge di Galileo:
\[
    \vec{PQ} + \vec{QR} = \vec{PR}
\]
\begin{center}
    \tikzset{external/export = true}
    \begin{tikzpicture}
        \draw[-Latex] (0,0) node [left] {$P$} -- (3,3) node [above] {$Q$};
        \draw[-Latex] (3,3) -- (5,1) node [right] {$R$};
        \draw[Green, -Latex] (0,0) -- (5,1);
    \end{tikzpicture}
    \tikzset{external/export = false}
\end{center}
\begin{teo}{Legge di Galileo}
    Per calcolare la somma di due vettori liberi con la \textbf{legge di Galileo} basta porre la coda del secondo vettore sulla testa del primo vettore e, per trovare il vettore somma, bisogna collegare la coda del primo vettore alla testa del secondo vettore.
\end{teo}

\subsection{Somma - Regola del parallelogramma}
Un modo alternativo di esprimere lo stesso concetto è tramite la regola del parallelogramma:
\begin{center}
    \tikzset{external/export = true}
    \begin{tikzpicture}
        \draw[-Latex] (0,0) node[left] {$P$} -- (3,3) node[above] {$Q$};
        \draw[-Latex] (0,0) -- (2,-1) node[below] {$S$};
        \draw[Blue, -Latex] (3,3) -- (5,2) node[right] {$R$};
        \draw[Blue, -Latex] (2,-1) -- (5,2);
        \draw[Green, -Latex] (0,0) -- (5,2);
    \end{tikzpicture}
    \tikzset{external/export = true}
\end{center}
\[
    \vec{PQ} + \vec{PS} = \vec{PR}
\]
\begin{teo}{Regola del parallelogramma}
    Per calcolare la somma di due vettori liberi con la \textbf{regola del parallelogramma}: si fanno coincidere le code dei due vettori; si duplica il primo vettore sulla testa del secondo e viceversa; le due teste si incontreranno nello stesso punto; collegando le code in comune e le teste in comune si otterrà il vettore somma.
\end{teo}

I due metodi sono totalmente equivalenti.

\subsection{Prodotto con scalare}
Con i vettori liberi il prodotto con scalare influisce sulla norma e su verso del vettore. In particolare:
\begin{teo}{Prodotto con scalare}
    Il \textbf{prodotto con scalare} sui vettori liberi è svolto nel seguente modo:
    \begin{itemize}
        \item \textbf{Norma:} Moltiplicata per il modulo del numero reale moltiplicato;
        \item \textbf{Direzione: } Invariata;
        \item \textbf{Verso: } Cambia se e solo se il numero con cui il vettore è moltiplicato è negativo
    \end{itemize}
\end{teo}
\begin{center}
    \tikzset{external/export = true}
    \begin{tikzpicture}
        \draw[-Latex] (0,0) -- node[above left]{$\vec v$} (1,1);
        \draw[-Latex] (1,0) -- node[above left]{$2\vec v$} (3,2);
        \draw[-Latex] (4,1) -- node[above left]{$-\vec v$} (3,0); 
    \end{tikzpicture}
    \tikzset{external/export = false}
\end{center}