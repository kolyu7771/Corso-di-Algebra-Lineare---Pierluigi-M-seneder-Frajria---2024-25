\chapter{Sistemi lineari}
\section{Introduzione}
I sistemi lineari non sono un argomento nuovo, ma sono una parte integrante del corso, quindi vale la pena riprendere l'argomento dall'inizio per stabilire una base comune.
\begin{newdef}{Sistema lineare}
    Un \textbf{sistema lineare} è un sistema di $n$ equazioni in $m$ incognite in cui il massimo grado presente è il primo.
    \[
        \begin{cases}
            a_{11}x_{1} + \dots + a_{1m}x_{m} = b_1 \\
            a_{21}x_{1} + \dots + a_{2m}x_{m} = b_2 \\
            \vdots \quad\quad \vdots \quad\quad \vdots \quad\quad \vdots \quad\quad \vdots \\
            a_{n1}x_{1} + \dots + a_{nm}x_{m} = b_n \\  
        \end{cases}
    \]
    $x_1, \dots, x_n$ sono le \textit{incognite}, $a_{11}, \dots, a_{nm}$ sono i \textit{coefficienti} e $b_1, \dots, b_n$ sono i \textit{termini noti}.
\end{newdef}

\begin{center}
    $x + y = 3$ è lineare \hspace{2cm} $xy + z = 3$ non è lineare \hspace{2cm} $y^2 + x = 0$ non è lineare \hspace{2cm} $e^x + 4y = 7$ non è lineare
\end{center}

\begin{newdef}{Sistema lineare omogeneo}
    Un sistema lineare è detto \textbf{omogeneo} se $b_1 = \dots = b_n = 0$, ovvero se è nella forma:
    \[
        \begin{cases}
            a_{11}x_{1} + \dots + a_{1m}x_{m} = 0 \\
            a_{21}x_{1} + \dots + a_{2m}x_{m} = 0 \\
            \vdots \quad\quad \vdots \quad\quad \vdots \quad\quad \vdots \quad\quad \vdots \\
            a_{n1}x_{1} + \dots + a_{nm}x_{m} = 0 \\  
        \end{cases}
    \]
\end{newdef}

La particolarità dei sistemi lineari è che un insieme di valori, per essere soluzione, deve soddisfare \textbf{\textit{contemporaneamente}} tutte e $n$ le equazioni.

\begin{newdef}{Soluzione di un sistema lineare}
    Una \textbf{soluzione} di un sistema lineare è un insieme $\{s_1, \dots, s_m\}$ di $m$ valori che, quando si pone $s_1 = x_1; \dots; s_m = x_m$, tutte le $n$ equazioni sono vere.

    Un sistema lineare può avere 0, 1 o infinite soluzioni.
\end{newdef}

\section{Metodi di risoluzione}
Si saranno già imparati diversi modi di risolvere sistemi di equazioni lineari. Si prenda per esempio il semplice sistema lineare:
\[
    \begin{cases}
        5x + 4y = -7 \\
        5x + 2y = -4
    \end{cases}
\]
Il metodo più basilare è quello per sostituzione, che con sistemi come questo è ammissibile, ma molto inefficace per sistemi con più variabili. Per risolvere un sistema lineare con questo metodo si riscrive una variabile in funzione delle altre e si cerca di semplificare man mano le equazioni fino a trovare il valore di una delle variabili e trovare di conseguenza quello delle altre. In pratica:
\[
    \begin{cases}
        5x + 4y = -7 \\
        5x = -4 - 2y
    \end{cases}
    \rightarrow
    \begin{cases}
        -4 - 2y + 4y = -7 \rightarrow 2y = -3\\
        5x = -4 - 2y \rightarrow x = -\frac{4}{5} - \frac{2}{5}y
    \end{cases}
\]
\[
    \begin{cases}
        y = -\frac{3}{2} \\
        x = -\frac{1}{5}
    \end{cases}
\]

\begin{exer}
    \textbf{Risolvere con il metodo di sostituzione i seguenti sistemi lineari:}

    $
        \begin{cases}
            6x + 3y = -6 \\
            4x + 6y = 6
        \end{cases}
    $;

    $
        \begin{cases}
            3x + 6y = 3 \\
            3x + 2y = -7
        \end{cases}
    $;

    $
        \begin{cases}
            12x - 6y = 6 \\
            3x + 12y = -2
        \end{cases}
    $;
\end{exer}

Un metodo generalmente preferibile è il metodo di riduzione. In sostanza si stratta di risolvere il sistema eliminando man mano le incognite dalle altre equazioni sfruttando il \textbf{principio di equivalenza dei sistemi lineari}.

Per rendere più chiare le idee: il primo principio di equivalenza dice che:
\[
    \text{Se } A = B \text{ allora } A + C = B + C
\]
Si prenda allora il seguente sistema di equazioni:
\[
    \begin{cases}
        \dots \\
        A = B \\
        C = D \\
        \dots
    \end{cases}
\]
Se $C = D$ allora $C$ e $D$ sono la stessa quantità, e perciò se $A = B$ è vero lo è anche $A + C = B + D$. Quindi si può sostituire ad $A = B$ la seconda equazione:
\[
    \begin{cases}
        \dots \\
        A + C = B + D \\
        C = D \\
        \dots
    \end{cases}
\]
Inoltre, per il secondo principio di equivalenza:
\[
    \text{Se } A = B \text{ allora } \alpha A = \alpha B
\]
Da cui si può dedurre che se $C = D$ allora $\beta C = \beta D$, e perciò, per lo stesso ragionamento di prima, si può dire che:
\[
    \begin{cases}
        \dots \\
        A + \beta C = B + \beta D \\
        C = D \\
        \dots
    \end{cases}
\]
\begin{nb}
    Non si può sostituire una qualsiasi equazione con la nuova equazione ricavata: si possono solo sostituire $A = B$ o $C = D$.
\end{nb}

Questa proprietà può essere sfruttata per \textit{eliminare} un'incognita per volta dalle equazioni, per arrivare poi a trovare il valore di una delle incognite e, di conseguenza, trovare quello delle altre.

Si prenda, a titolo d'esempio, il seguente sistema lineare di 3 equazioni e 3 incognite:
\[
    \begin{cases}
        4x + 3y + 6z = 2 \\
        x + 6y + 4z = 5 \\
        3x + 6y + 4z = -2
    \end{cases}
\]
Come abbiamo detto l'obiettivo è eliminare, una per una, tutte le incognite tranne una, di cui rimarrà semplicemente il valore.

La cosa più sensata da fare è sostituire alla seconda (o alla terza) equazione la differenza tra la seconda e la terza equazione, in quanto hanno entrambi gli stessi coefficienti per $y$ e $z$.
\[
    \begin{array}{c c c c}
        x & 6y & 4z & 5 \\
        -3x & -6y & -4z & 2 \\
        \hline
        -2x & 0 & 0 & 7
    \end{array}
\]
\[
    -2x = 7 \rightarrow x = -\frac{7}{2}
\]
Facendo un po' di ordine e sostituendo $x$ nelle altre due equazioni si ottiene il seguente sistema lineare:
\[
    \begin{cases}
        x = -\frac{7}{2} \\
        3y + 6z = 16 \\
        6y + 4z = \frac{17}{2}
    \end{cases}
\]
A questo punto, per eliminare $y$, si può sostituire alla terza equazione la differenza tra due volte la seconda equazione e la terza equazione:
\[
    \begin{array}{c c c}
        6y & 12z &32 \\
        -6y & -4z & -\frac{17}{2} \\
        \hline
        0 & 8z & \frac{47}{2}
    \end{array}
\]
\[
    8z = \frac{47}{2} \rightarrow z = \frac{47}{16}
\]
Quindi il nuovo sistema lineare è:
\[
    \begin{cases}
        x = -\frac{7}{2} \\
        3y + 6z = 16 \\
        z = \frac{47}{16}
    \end{cases}
\]
A questo punto trovare $y$ è banale:
\[
    3y + 3 \cdot \frac{47}{8} = 16 \rightarrow y + \frac{47}{8} = \frac{16}{3} \rightarrow y = -\frac{13}{24}
\]
\[
    \begin{cases}
        x = -\frac{7}{2} \\
        y = -\frac{13}{24} \\
        z = \frac{47}{16}
    \end{cases}
\]

\begin{exer}
    \text{Risolvere con il metodo di riduzione i seguenti sistemi lineari:}
    
    $
        \begin{cases}
            2x + 2y + 7z = 6 \\
            7x + 6y + z = 4 \\
            4x + 2y + 2z = -2
        \end{cases}
    $;

    $
        \begin{cases}
            x + 7y + 2z = 2 \\
            3x + 4y + 6z = -4 \\
            2x + 6y + 2z = 6
        \end{cases}
    $;

    $
        \begin{cases}
            6x + 6y + 7z = 6 \\
            6x + 6y + 6z = 2 \\
            4x + y + 6z = 6
        \end{cases}
    $;
\end{exer}

Esistono altri metodi di risoluzione, che però per la maggior parte verranno ignorati. L'altro metodo che verrà esplorato più avanti, nella parte astratta, è il metodo di Cramer.

\section{Sistemi come matrici e vettori}
Riprendendo il discorso teorico sui sistemi lineari, questi si possono rappresentare tramite matrici e vettori.
\begin{newdef}{Rappresentazione matriciale dei sistemi lineari}
    Un sistema lineare generico può essere rappresentato come:
    \[
        A\textbf{x} = \textbf{b}
    \]
    dove $A$ è la matrice dei coefficienti:
    \[
        A =
        \begin{sqmatrix}{c c c c}
            a_{11} & a_{12} & \dots & a_{1m} \\
            a_{21} & a_{22} & \dots & a_{2m} \\
            \vdots & \vdots & \ddots & \vdots \\
            a_{n1} & a_{n2} & \dots & a_{nm} \\
        \end{sqmatrix}
    \]
    \textbf{x} è il vettore delle incognite:
    \[
        \textbf{x} =
        \begin{sqcolvec}
            x_1 \\
            x_2 \\
            \vdots \\
            x_m
        \end{sqcolvec}
    \]
    E \textbf{b} è il vettore dei termini noti:
    \[
        \textbf{b} =
        \begin{sqcolvec}
            b_1 \\
            b_2 \\
            \vdots \\
            b_n
        \end{sqcolvec}
    \]
\end{newdef}

Quindi, nel caso particolare del sistema lineare:
\[
    \begin{cases}
        4x + 3y + 6z = 2 \\
        x + 6y + 4z = 5 \\
        3x + 6y + 4z = -2
    \end{cases}
\]
Questo diventerebbe:
\[
    \begin{sqmatrix}{c c c}
        4 & 3 & 6 \\
        1 & 6 & 4 \\
        3 & 6 & 4
    \end{sqmatrix}
    \begin{sqcolvec}
        x \\
        y \\
        z
    \end{sqcolvec}
    =
    \begin{sqcolvec}
        2 \\
        5 \\
        -2
    \end{sqcolvec}
\]
Da qui si può vedere il motivo della definizione di prodotto riga per colonna data all'inizio. Infatti, svolgendolo, si riottengono le equazioni del sistema iniziale:
\[
    \begin{sqmatrix}{c c c}
        4 & 3 & 6 \\
        1 & 6 & 4 \\
        3 & 6 & 4
    \end{sqmatrix}
    \begin{sqcolvec}
        x \\
        y \\
        z
    \end{sqcolvec}
    =
    \begin{sqcolvec}
        4x + 3y + 6z \\
        x + 6y + 4z \\
        3x + 6y + 4z
    \end{sqcolvec}
\]

\begin{newdef}{Matrice completa di un sistema lineare}
    Dato un certo sistema lineare, la \textbf{matrice completa} del sistema è la matrice dei coefficienti a cui viene affiancato il vettore colonna dei termini noti.
    \[
        (A|\textbf{b}) =
        \begin{sqmatrix}{c c c c c}
            a_{11} & a_{12} & \dots & a_{1m} & b_{1} \\
            a_{21} & a_{22} & \dots & a_{2m} & b_{2} \\
            \vdots & \vdots & \ddots & \vdots & \vdots \\
            a_{n1} & a_{n2} & \dots & a_{nm} & b_{n}\\
        \end{sqmatrix}
    \]
\end{newdef}

\section{Metodo di Eliminazione di Gauss (MEG)}
Con questa nuova rappresentazione è possibile definire azioni che non cambiano l'insieme delle soluzioni di un sistema lineare anche sulle matrici. Queste azioni sono dette \textbf{operazioni elementari di riga}.
\begin{newdef}{Operazioni elementari di riga}
    Le \textbf{operazioni elementari di riga} sono operazioni fatte sulle righe che non cambiano l'insieme delle soluzioni del sistema lineare associato. Sono:
    \begin{enumerate}
        \item \textbf{Scambio} di due righe;
        \item \textbf{Prodotto con scalare} di una riga;
        \item \textbf{Somma tra multipli} di due righe.
    \end{enumerate}
    Per indicare lo svolgimento di queste operazioni si utilizza il simbolo $\sim$.
\end{newdef}

L'obiettivo del MEG è quello di portare la matrice completa del sistema una forma detta \text{a gradini}
\begin{newdef}{Matrice a gradini}
    Una matrice $n \times m$ è detta \textbf{a gradini} se il \textit{pivot} (primo elemento non nullo) di ogni riga è in una colonna più a destra di quello della riga precedente.
    \[
        \begin{sqmatrix}{c c c c c}
            * & * & * & * & * \\
            0 & * & * & * & * \\
            0 & 0 & * & * & * \\
            0 & 0 & 0 & * & *
        \end{sqmatrix}
    \]
\end{newdef}
\begin{nb}
    Non tutte le \textit{colonne} hanno necessariamente un pivot.
\end{nb}
Il motivo per cui si punta ad arrivare a una matrice a gradini è che questa corrisponde a un sistema ``ridotto'', ovvero in cui è stato applicato il metodo di riduzione in maniera completa. Per esempio:
\[
    \begin{sqmatrix}{c c c c c}
        7 & 3 & 1 & 4 & 3 \\
        0 & 9 & 1 & 5 & 0 \\
        0 & 0 & 3 & 1 & 10 \\
        0 & 0 & 0 & 5 & 2
    \end{sqmatrix}
    \text{ equivale a }
    \begin{cases}
        7x + 3y + z + 4w = 3 \\
        9y + z + 5w = 0 \\
        3z + w = 10 \\
        5w = 2
    \end{cases}
\]
A questo punto non resta che introdurre effettivamente il MEG:

\begin{teo}{Metodo di Eliminazione di Gauss}
    Il \textbf{Metodo di Eliminazione di Gauss} è un metodo di riduzione di una matrice nella sua forma a gradini tramite operazioni elementari di riga. Il metodo è il seguente:
    \begin{enumerate}
        \item Attraverso lo scambio di righe, portare la matrice a non avere righe con pivot più a sinistra dei pivot delle righe precedenti, avendo cura di mettere eventuali righe nulle in fondo alla matrice;
        \item Individuare il pivot della riga in analisi (inizialmente la prima) ed eliminare attraverso la somma tra multipli di righe tutti gli elementi della colonna corrispondente sottostanti il pivot;
        \item Ripetere il processo per la riga successiva fino ad avere una matrice a gradini.
    \end{enumerate}
\end{teo}
Una volta ottenuta una matrice a gradini si può risolvere il sistema associato alla matrice sostituendo man mano i valori trovati.
\begin{esempio}
    \textbf{Risolvere il seguente sistema lineare usando il MEG:}
    \[
        \begin{cases}
            3x + 5y + 3z + 4w = 1 \\
            6x + 5y + 3z + 5w = -5 \\
            3x + 2y + 3z + 4w = 4 \\
            6x + 5y + 6z + 4w = 5
        \end{cases}
    \]
    Si inizia scrivendo la matrice completa del sistema:
    \[
        (A|\textbf{b}) =
        \begin{sqmatrix}{c c c c c}
            3 & 5 & 3 & 4 & 1 \\
            6 & 5 & 3 & 5 & -5 \\
            3 & 2 & 3 & 4 & 4 \\
            6 & 5 & 6 & 4 & 5
        \end{sqmatrix}
    \]
    Qui sotto verranno scritti i vari passaggi del MEG svolti:
    \begin{center}
        \begin{minipage}{.4\textwidth}
            \[
                \begin{sqmatrix}{c c c c c}
                    3 & 5 & 3 & 4 & 1 \\
                    0 & 5 & 3 & 3 & 7 \\
                    0 & 3 & 0 & 0 & -3 \\
                    0 & 5 & 0 & 4 & -3
                \end{sqmatrix}
            \]
        \end{minipage}
        \begin{minipage}{.4\textwidth}
            \[
                R_2 = 2R_1 - R_2
            \]
            \[
                R_3 = R_1 - R_3
            \]
            \[
                R_4 = 2R_1 - R_4
            \]
        \end{minipage}
    \end{center}
    \begin{center}
        \begin{minipage}{.4\textwidth}
            \[
                \begin{sqmatrix}{c c c c c}
                    3 & 5 & 3 & 4 & 1 \\
                    0 & 5 & 3 & 3 & 7 \\
                    0 & 0 & 3 & 3 & 12 \\
                    0 & 0 & 3 & -1 & 10
                \end{sqmatrix}
            \]
        \end{minipage}
        \begin{minipage}{.4\textwidth}
            \[
                R_3 = R_2 - \frac{5}{3}R_3
            \]
            \[
                R_4 = R_2 - R_4
            \]
        \end{minipage}
    \end{center}
    \begin{center}
        \begin{minipage}{.4\textwidth}
            \[
                \begin{sqmatrix}{c c c c c}
                    3 & 5 & 3 & 4 & 1 \\
                    0 & 5 & 3 & 3 & 7 \\
                    0 & 0 & 3 & 3 & 12 \\
                    0 & 0 & 0 & 4 & 2
                \end{sqmatrix}
            \]
        \end{minipage}
        \begin{minipage}{.4\textwidth}
            \[
                R_4 = R_3 - R_4
            \]
        \end{minipage}
    \end{center}
    Ora che la matrice è a gradini si può risolvere il sistema associato:
    \[
        \begin{cases}
            3x + 5y + 3z + 4w = 1 \\
            5y + 3z + 3w = 7 \\
            3z + 3w = 12 \\
            4w = 2 \rightarrow w = \frac{1}{2}
        \end{cases}
    \]
    \[
        3z + \frac{3}{2} = 12 \rightarrow z = \frac{7}{2}
    \]
    \[
        5y + \frac{21}{2} + \frac{3}{2} = 7 \rightarrow y = -1
    \]
    \[
        3x - 5 + \frac{21}{2} + 2 = 1 \rightarrow x = -\frac{13}{6}
    \]
    Il vettore soluzione sarà:
    \[
        \textbf{x} =
        \begin{sqcolvec}
            x \\
            y \\
            z \\
            w
        \end{sqcolvec}
        =
        \begin{sqcolvec}
            -\frac{13}{6} \\
            -1 \\
            \frac{7}{2} \\
            \frac{1}{2}
        \end{sqcolvec}
    \]
\end{esempio}
\begin{exer}
    \textbf{Risolvere i seguenti sistemi lineari con il MEG:}

    $
        \begin{cases}
            -4x + 10y - z = -6 \\
            -12x + 2y + 2z = -5 \\
            9x + 10y + 2z = 3
        \end{cases}
    $;

    $
        \begin{cases}
            -7x + 4y - 11z = 3 \\
            -6x - 2y + 2z = -1 \\
            5x - 4y - 11z = -5
        \end{cases}
    $;

    $
        \begin{cases}
            6x + 6y + 4z + 2w = -7 \\
            2x + 2y + 5z + 7w = -4 \\
            2x + 6y + 6z + 4w = 5 \\
            4v + 2w + 4z + 4w = -7
        \end{cases}
    $;
\end{exer}