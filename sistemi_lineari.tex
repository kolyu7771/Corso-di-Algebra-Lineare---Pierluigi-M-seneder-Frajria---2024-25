\chapter{Sistemi lineari}
\section{Introduzione}
I sistemi lineari non sono un argomento nuovo, ma sono una parte integrante del corso, quindi vale la pena riprendere l'argomento dall'inizio per stabilire una base comune.
\begin{newdef}{Sistema lineare}
    Un \textbf{sistema lineare} è un sistema di $n$ equazioni in $m$ incognite in cui il massimo grado presente è il primo.
    \[
        \begin{cases}
            a_{11}x_{1} + \dots + a_{1m}x_{m} = b_1 \\
            a_{21}x_{1} + \dots + a_{2m}x_{m} = b_2 \\
            \vdots \quad\quad \vdots \quad\quad \vdots \quad\quad \vdots \quad\quad \vdots \\
            a_{n1}x_{1} + \dots + a_{nm}x_{m} = b_n \\  
        \end{cases}
    \]
    $x_1, \dots, x_n$ sono le \textit{incognite}, $a_{11}, \dots, a_{nm}$ sono i \textit{coefficienti} e $b_1, \dots, b_n$ sono i \textit{termini noti}
\end{newdef}
\begin{center}
    $x + y = 3$ è lineare \hspace{2cm} $xy + z = 3$ non è lineare \hspace{2cm} $y^2 + x = 0$ non è lineare \hspace{2cm} $e^x + 4y = 7$ non è lineare
\end{center}

La particolarità dei sistemi lineari è che un insieme di valori, per essere soluzione, deve soddisfare \textbf{\textit{contemporaneamente}} tutte e $n$ le equazioni.

\begin{newdef}{Soluzione di un sistema lineare}
    Una \textbf{soluzione} di un sistema lineare è un insieme $\{s_1, \dots, s_m\}$ di $m$ valori che, quando si pone $s_1 = x_1; \dots; s_m = x_m$, tutte le $n$ equazioni sono vere.

    Un sistema lineare può avere 0, 1 o infinite soluzioni.
\end{newdef}

\section{Metodi di risoluzione}
Si saranno già imparati diversi modi di risolvere sistemi di equazioni lineari. Si prenda per esempio il semplice sistema lineare:
\[
    \begin{cases}
        5r + 4s = -7 \\
        5r + 2s = -4
    \end{cases}
\]
Il metodo più basilare è quello per sostituzione, che con sistemi come questo è ammissibile, ma molto inefficace per sistemi con più variabili. Per risolvere un sistema lineare con questo metodo si riscrive una variabile in funzione delle altre e si cerca di semplificare man mano le equazioni fino a trovare il valore di una delle variabili e trovare di conseguenza quello delle altre. In pratica:
\[
    \begin{cases}
        5r + 4s = -7 \\
        5r = -4 - 2s
    \end{cases}
    \rightarrow
    \begin{cases}
        -4 - 2s + 4s = -7 \rightarrow 2s = -3\\
        5r = -4 - 2s \rightarrow r = -\frac{4}{5} - \frac{2}{5}s
    \end{cases}
\]
\[
    \begin{cases}
        s = -\frac{3}{2} \\
        r = -\frac{1}{5}
    \end{cases}
\]

Un metodo generalmente preferibile è il metodo di riduzione. In sostanza si stratta di risolvere il sistema eliminando man mano le incognite dalle altre equazioni sfruttando il \textbf{principio di equivalenza dei sistemi lineari}.

Per rendere più chiare le idee: il primo principio di equivalenza dice che:
\[
    \text{Se } A = B \text{ allora } A + C = B + C
\]
Si prenda allora il seguente sistema di equazioni:
\[
    \begin{cases}
        \dots \\
        A = B \\
        C = D \\
        \dots
    \end{cases}
\]
Se $C = D$ allora $C$ e $D$ sono la stessa quantità, e perciò se $A = B$ è vero lo è anche $A + C = B + D$. Quindi si può sostituire ad $A = B$ la seconda equazione:
\[
    \begin{cases}
        \dots \\
        A + C = B + D \\
        C = D \\
        \dots
    \end{cases}
\]
Inoltre, per il secondo principio di equivalenza:
\[
    \text{Se } A = B \text{ allora } \alpha A = \alpha B
\]
Da cui si può dedurre che se $C = D$ allora $\beta C = \beta D$, e perciò, per lo stesso ragionamento di prima, si può dire che:
\[
    \begin{cases}
        \dots \\
        A + \beta C = B + \beta D \\
        C = D \\
        \dots
    \end{cases}
\]

Questa proprietà può essere sfruttata per \textit{eliminare} un'incognita per volta dalle equazioni, per arrivare poi a trovare il valore di una delle incognite e, di conseguenza, trovare quello delle altre.

Si prenda, a titolo d'esempio, il seguente sistema lineare di 3 equazioni e 3 incognite:
\[
    \begin{cases}
        4x + 3y + 6z = 2 \\
        x + 6y + 4z = 5 \\
        3x + 6y + 4z = -2
    \end{cases}
\]
